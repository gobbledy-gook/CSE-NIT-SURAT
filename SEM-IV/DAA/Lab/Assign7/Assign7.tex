\documentclass{article}
\usepackage{listings}
\usepackage{graphicx}
\usepackage{color}
\usepackage{pgfplots}
\usepackage{tabularx}
\usepackage{float}
\usepackage{multirow}
\pgfplotsset{compat=1.18}
\usepackage[margin=0.75in]{geometry}
\definecolor{dkgreen}{rgb}{0,0.6,0}
\definecolor{gray}{rgb}{0.5,0.5,0.5}
\definecolor{mauve}{rgb}{0.58,0,0.82}
\lstset{,
  language=C++,
  aboveskip=3mm,
  belowskip=3mm,
  showstringspaces=false,
  columns=flexible,
  basicstyle={\small\ttfamily},
  numbers=none,
  numberstyle=\tiny\color{gray},
  keywordstyle=\color{blue},
  commentstyle=\color{dkgreen},
  stringstyle=\color{mauve},
  breaklines=true,
  breakatwhitespace=true,
  tabsize=3
}
\title{Assignment 7}
\author{Bhavya[U21CS100], Garvit[U21CS089], Swayam[U21CS079]}
\begin{document}
    \maketitle

    \section{Hardware Specifications}

    \noindent\makebox[\textwidth] {
    \begin{tabularx}{\textwidth} {XX}
    \hline
    Machine Type & Laptop
\\ \hline
    CPU & 11th Gen Intel(R) Core(TM) i5-1135G7
\\ \hline
    Frequency & 2.40 GHz
\\ \hline
    L1 Cache & 320 KB
\\ \hline 
    L2 Cache & 5.0 MB
\\ \hline
    L3 Cache & 8.0 MB
\\ \hline 
    RAM & 16 GB
\\ \hline
    Operating System & Windows 11 Home
\\ \hline
    Implementation Language & C++
\\ \hline
    Compiler & g++ 12.1.0
\\ \hline
    Libaries Used & 
    \noindent\makebox[\linewidth] {

    \begin{tabularx}{\linewidth} {X}
    iostream
\\ 
    chrono
\\
    vector
    \end{tabularx}
}
\\ \hline
\end{tabularx}
}
    \section{Buy and Sell Stock}
    \subsection{Question}
    The following assignment you can do as a group (maximum size: 4 students) or indi-
vidually. Only one student from the group should upload the assignment at Google
\\
classroom. On the first page of submission, you have to clearly mention the names and
roll numbers of students of the group. The assessment will be based on the novelty
of the problem (how different the computation problem is from problems submitted
by other groups.), how interesting the problem is, how elegantly you have written the
problem statement, pseudocode, and analysis, how elegantly you implemented and
analyzed the algorithms, and the viva voce.

    \subsection{Input Format:}
        Matrix Size - $n$\\
        Position of the Statue - $x$ $y$\\
    \\


     \newpage
     \section{Solution}
     \subsection{Algorithm}
     \begin{enumerate}
        \item Define a recursive function named c that takes four arguments:
           a. i, an integer representing the current index in the prices list.
           b. buy, an integer representing whether the current state is a "buy" state (1) or "not buy" state (0).
           c. prices, a vector of integers representing the stock prices at each index.
           d. dp, a 2D vector of integers representing the memoization table.
        
           The function returns an integer representing the maximum profit that can be earned from the current state.
        
        \item Check if the index i is greater than or equal to the length of the prices list.
           a. If yes, return 0 as there are no more stocks left to buy or sell.
        
        \item Check if the memoization table dp at the index [i][buy] is not equal to -1.
           a. If yes, return the value stored in the memoization table as the result.
        
        \item Define a variable ans to store the maximum profit that can be earned from the current state.
        
        \item Check if the current state is a "buy" state (buy = 1).
           a. If yes, set ans to the maximum value of:
              i. -1 times the stock price at index i, added to the result of the c function called with the arguments (i+1, 0, prices, dp).
                 This represents the case where the stock is bought at index i, and then sold later at some point after i+1.
              ii. The result of the c function called with the arguments (i+1, 1, prices, dp).
                 This represents the case where the stock is not bought at index i, and the state remains a "buy" state.
        
           b. If no, set ans to the maximum value of:
              i. The stock price at index i, added to the result of the c function called with the arguments (i+1, 1, prices, dp).
                 This represents the case where the stock is sold at index i, and then the state changes to a "not buy" state.
              ii. The result of the c function called with the arguments (i+1, 0, prices, dp).
                 This represents the case where the stock is not sold at index i, and the state remains a "not buy" state.
        
        \item Store the value of ans in the memoization table dp at the index [i][buy].
        
        \item Return ans as the result of the function.
        
        \item Define a function named maxProfit that takes one argument:
           a. prices, a vector of integers representing the stock prices at each index.
        
           The function returns an integer representing the maximum profit that can be earned from buying and selling stocks based on the given prices.
        
        \item Create a 2D vector dp of size n x 2, where n is the length of the prices vector, and initialize all elements to -1.
        
        \item Call the c function with the arguments (0, 1, prices, dp), which represents starting at the first index with a "buy" state.
        
        \item Return the value returned by the c function as the result of the maxProfit function.
        
        \item Define a vector named prices with some initial stock prices.
        
        \item Create an instance of the Solution class named s.
        \item Call the maxProfit function with the prices vectore as an argument, and store the result in a variable named max-profit
        \item Print the string "Max Profit:" followed by the value of the max-profit variable.
        
     \end{enumerate}
     
     \newpage
     \subsection{Psuedo Code}{Greedy Approach}

     \begin{lstlisting}
        function c(i: int, buy: int, prices: List[int], dp: List[List[int]]) -> int:
        if i >= len(prices):
            return 0
        if dp[i][buy] != -1:
            return dp[i][buy]
        ans: int
        if buy:
            ans = max(-1 * prices[i] + c(i + 1, 0, prices, dp), c(i + 1, 1, prices, dp))
        else:
            ans = max(prices[i] + c(i + 1, 1, prices, dp), c(i + 1, 0, prices, dp))
        dp[i][buy] = ans
        return ans

    function maxProfit(prices: List[int]) -> int:
        dp: List[List[int]] = [[-1 for _ in range(2)] for _ in range(len(prices))]
        return c(0, 1, prices, dp)

prices: List[int] = [7, 1, 5, 3, 6, 4]
s: Solution = Solution()
max_profit: int = s.maxProfit(prices)
print("Max Profit:", max_profit)
       \end{lstlisting} 
    
    \newpage
    \subsection{C++ Code}
    \begin{lstlisting}
        #include <iostream>
        #include <vector>
        
        using namespace std;
        
        class Solution {
        public:
            int c(int i, int buy, vector<int>& prices, vector<vector<int>>& dp) {
                if (i >= prices.size()) {
                    return 0;
                }
                int ans;
                if (dp[i][buy] != -1) {
                    return dp[i][buy];
                }
                if (buy) {
                    ans = max(-1 * prices[i] + c(i + 1, 0, prices, dp), c(i + 1, 1, prices, dp));
                }
                else {
                    ans = max(prices[i] + c(i + 1, 1, prices, dp), c(i + 1, 0, prices, dp));
                }
                dp[i][buy] = ans;
                return ans;
            }
        
            int maxProfit(vector<int>& prices) {
                vector<vector<int>> dp(prices.size(), vector<int>(2, -1));
                return c(0, 1, prices, dp);
            }
        };
        
        int main() {
            vector<int> prices = {7, 1, 5, 3, 6, 4};
            Solution s;
            int maxProfit = s.maxProfit(prices);
            cout << "Max Profit: " << maxProfit << endl;
            return 0;
        }
    \end{lstlisting}

\end{document}